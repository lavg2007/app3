% ======================================================================


% Sujet du document
% Informations importantes
%
%
% Prénom Nom
% H. Dube 2019
% ======================================================================
% Ce code rassemble les efforts d'étudiants de la faculté de génie  
% de l'université de Sherbrooke afin de faire un template LaTeX moderne
% dédié à l'écriture de rapport universitaire.
% Ce document est libre d'être utilisé et modifié.
% ======================================================================

% ----------------------------------------------------
% Initialisation
% ----------------------------------------------------
\documentclass{udes_rapport} % Voir udes_rapport.cls

\begin{document}
\selectlanguage{french}

% ----------------------------------------------------
% Configurer la page titre
% ----------------------------------------------------

% Information
\faculte{Génie}
\departement{génie électrique et génie informatique}
\app{3}{Modélisation et identification de systèmes dynamiques}
\professeur{M. Raef Cherif et M. Jean-Samuel Lauzon}
\etudiants{Hubert Dubé - dubh3401 \\ Gabriel Lavoie - lavg2007}
\dateRemise{2 octobre 2019}


% ======================================================================
\pagenumbering{roman} % met les numéros de pages en romain
% ----------------------------------------------------
% Page titre
% ----------------------------------------------------
\fairePageTitre{LOGO} % Options: [STD, LOGO]
\newpage

% ----------------------------------------------------
% Table des matières
% ----------------------------------------------------
\tableofcontents
\newpage


% ----------------------------------------------------
% Table des figures
% ----------------------------------------------------
\listoffigures
\newpage



% ======================================================================
% Document
% ======================================================================
\pagenumbering{arabic} % met des chiffres arabes
\setcounter{page}{1} % reset les numéros de pages
%%%%%%%%%%%%%%%%%%%%%%%%%%%%%%%%%%%%%%%%%%%%%%%%%%%%%%
%{Analyse du signal
%%%%%%%%%%%%%%%%%%%%%%%%%%%%%%%%%%%%%%%%%%%%%%%%%%%%%%
\section{Introduction}

\section{Fonction de transfert en boucle ouverte et fermée}
Les équations des parties précedente permette d'obtenir toutes les fonctions de transfert nécessaire pour représenter le système en utilisant uniquement des système d'ordre 1. De manière analytique, en rerpenant l'équation (XXXXX):
\[	G0 = \frac{e_a}{e_{in}} = \frac{K/T}{s +1/T}	\]
avec l'équation (XXXXX):
\[	G1 = \frac{I_a}{e_a - \frac{k_b \omega_L}{N}} = \frac{1}{L_a s + R_a}						\]
avec l'équation (XXXXX):
\[	G2 = \frac{\omega_L}{I_a} = \frac{\frac{k_i}{\frac{J_m}{N}+N J_L}}{s + \frac{\frac{B_m}{N} + N B_L}{\frac{J_m}{N}+N J_L}}	\]
adin d'obtenir $\theta _L$ comme sortie il suffit d'ajouter un intégrateur à la suite de $G2$:
\[	G4 = \frac{\theta _L}{\omega _L} = \frac{1}{s}	\]
\subsection{Représentation graphique}
Sous forme de schéma bloc, le système peut être représenté à la figure \ref{fig:bloc}.
\insertFigure{open_closed_loop}{0.7}{Schéma bloc du système}{fig:bloc}
La partie en noire correspond au circuit en boucle ouverte et avec l'ajout de la partie en rouge, il s'agit du circuit en boucle fermé. Le même schéma peut être obtenu sous forme de graphe de fluence à la figure \ref{fig:fluence}.
\insertFigure{graphe_fluence}{0.7}{Graphe de fluence du système}{fig:fluence}
Encore une fois, la boucle ouverte est en noir et la boucle fermée est en rouge. Dans les deux figures, les fonctions $G$ représentent les fonctions d'ordre 1 déclarées au début de cette section.
\subsection{Représentation en fonction de transfert}
Afin d'obtenir la fonction de transfert en boucle ouverte (FTBO) la loi de Masson est appliqué sur la partie noire de la figure \ref{fig:bloc}. De manière analytique, la FTBO peut être représentée comme :
\begin{equation}
M = \frac{\theta_L}{e_{in}} = \frac{G_0 G_1 G_2 G_4}{1 + G_1 G_2 G_3}
\label{eq:FTBO}
\end{equation}
Pour obtenir la fonction de transfert en boucle fermé (FTBF) il faut inclure l'équation (XXXX) dans \eqref{eq:FTBO}, car elle permet de lier l'entrée de la consigne à la sortie du système. En effet,
\[	M = \frac{\theta_L}{k_p (\theta_L -\theta_L )}	\]
En manipulant l'équation on obtient donc
\begin{equation}
H = \frac{\theta_L}{\theta_d} = \frac{M \cdot k_p}{1+M \cdot k_p}
\label{eq:FTBF}
\end{equation}
En remplacant les paramètres de \eqref{eq:FTBO} et \eqref{eq:FTBF} et en normalisant la FTBF peut être écrite comme:
\begin{equation}
H = \frac{\theta_L}{\theta_d} = \frac{6.625e05}{s^4 + 1101s^3 + 1.018e05s^2 + 1.708e05s + 6.625e05}
\label{eq:nmum_FTBF}
\end{equation}
Cette solution est obtenue à l'aide de Matlab vue la complexité de l'équation.
\section{Réduction}
\section{Identification du moteur}
Les résultats de l'expérience à rotor bloqué permettent d'identifier les valeurs de $R_A$, $k_i$ et $k_b$. Durant cet expérience, la force contre électromotrice est nulle (il n'y a pas de rotation) et l'hypothèse est tel que l'inductance est beaucoup plus petite que la résistance. Ceci nous permet d'obtenir :
\[	R_a = \frac{e_a}{I_a} = 7.34 \Omega	\]
\[	k_i = \frac{T_m}{I_a} =	0.48 		\]
\[	k_b = k_i							\]
Ensuite, afin de trouver les deux dernières variables, $B_m$ et $J_m$, il est possible de passer par la fonction de transfert qu'on obtient par l'équation physique du couple exercé par le moteur. Partant de 
\begin{equation}
k_i I_a = J_m \dot{\omega} _m + B_m \omega _m
\label{eq:torque}
\end{equation}
et en utilisant la même hypothèse que dans la partie précedente, la fonction qui sera obtenue sera d'ordre 1 seulement. Avec l'équation de $I_a$ obtenue précedement,où l'hypothese permet d'éliminer son élément différentiel, la fonction de transfert de la vitesse de rotation du moteur en fonction de la tension de l'armature est:
\begin{equation}
\frac{\omega_m}{e_a} = \frac{k_i/R_a}{J_m s + B_m + \frac{k_i k_b}{R_a}}
\label{eq:tf_moteur}
\end{equation}

Comme il est maintenant possible d'évaluer le rapport entre la vitesse angulaire et la tension du moteur par un système d'ordre 1. L'équation d'un tel systeme peut être représenté par l'équation générale suivante :
\begin{equation}
tf = \frac{\omega_m}{e_a} = \frac{K}{1+Ts}
\label{eq:tf_odr1}
\end{equation}
En utilisant la méthode des moindres carrée et les données fournies, les valeurs du gain et de la constante de temps peuvent être obtenue pour le modèle simplifié. Les valeurs suivantes sont ressorties en utilisant Matlab :
\[	K = 1.4121	\]
\[	T = 0.4984	\]

Réutilisant les equations \eqref{eq:tf_odr1} et \eqref{eq:tf_moteur} avec les résultats pour K et T, il est possible de trouver les valeurs de $J_m$ et $B_m$ pour finaliser l'identificaiton du moteur.
\[B_m = 0.0150	\]
\[J_m = 0.0229	\]
\section{Linéarisation}

\section{Conclusion}


\begin{comment}
\begin{center}
	\centering
	\includegraphics[width=0.7\textwidth]{puissance}
	\captionof{figure}{Spectre de puissance d'une onde de 1kHz}
	\label{puissance}
\end{center}


\section{Filtres FIR}
\noindent\begin{minipage}{\textwidth} 
\begin{minipage}{0.5\textwidth}
  \centering
  \includegraphics[width=.75\linewidth]{ampFIR}
  \captionof{subfigure}{Amplitude}
  \label{FIR:ampFIR}
\end{minipage}%
\begin{minipage}{0.5\textwidth}
  \centering 
  \includegraphics[width=.75\linewidth]{phaseCute} 
  \captionof{subfigure}{Phase} 
  \label{FIR:phaseFIR} 
\end{minipage} 
\captionof{figure}{Filtre IIR} 
\label{FIR} 
\end{minipage}
\end{comment}

\end{document}













