% ======================================================================


% Sujet du document
% Informations importantes
%
%
% Prénom Nom
% H. Dube 2019
% ======================================================================
% Ce code rassemble les efforts d'étudiants de la faculté de génie  
% de l'université de Sherbrooke afin de faire un template LaTeX moderne
% dédié à l'écriture de rapport universitaire.
% Ce document est libre d'être utilisé et modifié.
% ======================================================================

% ----------------------------------------------------
% Initialisation
% ----------------------------------------------------
\documentclass{udes_rapport} % Voir udes_rapport.cls

\begin{document}
\selectlanguage{french}

% ----------------------------------------------------
% Configurer la page titre
% ----------------------------------------------------

% Information
\faculte{Génie}
\departement{génie électrique et génie informatique}
\app{1}{Éléments de statique et de dynamique}
\professeur{M. Raef Cherif et M. Jean-Samuel Lauzon}
\etudiants{Hubert Dubé - dubh3401 \\ Marc Sirois - sirm2508\\ Gabriel Lavoie - lavg2007}
\dateRemise{18 septembre 2019}


% ======================================================================
\pagenumbering{roman} % met les numéros de pages en romain
% ----------------------------------------------------
% Page titre
% ----------------------------------------------------
\fairePageTitre{LOGO} % Options: [STD, LOGO]
\newpage

% ----------------------------------------------------
% Table des matières
% ----------------------------------------------------
\tableofcontents
\newpage


% ----------------------------------------------------
% Table des figures
% ----------------------------------------------------
\listoffigures
\newpage



% ======================================================================
% Document
% ======================================================================
\pagenumbering{arabic} % met des chiffres arabes
\setcounter{page}{1} % reset les numéros de pages
%%%%%%%%%%%%%%%%%%%%%%%%%%%%%%%%%%%%%%%%%%%%%%%%%%%%%%
%{Analyse du signal
%%%%%%%%%%%%%%%%%%%%%%%%%%%%%%%%%%%%%%%%%%%%%%%%%%%%%%
\section{Introduction}

\section{Conclusion}


\begin{comment}
\begin{center}
	\centering
	\includegraphics[width=0.7\textwidth]{puissance}
	\captionof{figure}{Spectre de puissance d'une onde de 1kHz}
	\label{puissance}
\end{center}


\section{Filtres FIR}
\noindent\begin{minipage}{\textwidth} 
\begin{minipage}{0.5\textwidth}
  \centering
  \includegraphics[width=.75\linewidth]{ampFIR}
  \captionof{subfigure}{Amplitude}
  \label{FIR:ampFIR}
\end{minipage}%
\begin{minipage}{0.5\textwidth}
  \centering 
  \includegraphics[width=.75\linewidth]{phaseCute} 
  \captionof{subfigure}{Phase} 
  \label{FIR:phaseFIR} 
\end{minipage} 
\captionof{figure}{Filtre IIR} 
\label{FIR} 
\end{minipage}
\end{comment}

\end{document}













